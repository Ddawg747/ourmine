% This is "sig-alternate.tex" V1.9 April 2009
% This file should be compiled with V2.4 of "sig-alternate.cls" April 2009
%
% This example file demonstrates the use of the 'sig-alternate.cls'
% V2.4 LaTeX2e document class file. It is for those submitting
% articles to ACM Conference Proceedings WHO DO NOT WISH TO
% STRICTLY ADHERE TO THE SIGS (PUBS-BOARD-ENDORSED) STYLE.
% The 'sig-alternate.cls' file will produce a similar-looking,
% albeit, 'tighter' paper resulting in, invariably, fewer pages.
%
% ----------------------------------------------------------------------------------------------------------------
% This .tex file (and associated .cls V2.4) produces:
%       1) The Permission Statement
%       2) The Conference (location) Info information
%       3) The Copyright Line with ACM data
%       4) NO page numbers
%
% as against the acm_proc_article-sp.cls file which
% DOES NOT produce 1) thru' 3) above.
%
% Using 'sig-alternate.cls' you have control, however, from within
% the source .tex file, over both the CopyrightYear
% (defaulted to 200X) and the ACM Copyright Data
% (defaulted to X-XXXXX-XX-X/XX/XX).
% e.g.
% \CopyrightYear{2009} %will cause 2007 to appear in the copyright line.
% \crdata{0-12345-67-8/90/12} will cause 0-12345-67-8/90/12 to appear in the copyright line.
%
% ---------------------------------------------------------------------------------------------------------------
% This .tex source is an example which *does* use
% the .bib file (from which the .bbl file % is produced).
% REMEMBER HOWEVER: After having produced the .bbl file,
% and prior to final submission, you *NEED* to 'insert'
% your .bbl file into your source .tex file so as to provide
% ONE 'self-contained' source file.
%
% ================= IF YOU HAVE QUESTIONS =======================
% Questions regarding the SIGS styles, SIGS policies and
% procedures, Conferences etc. should be sent to
% Adrienne Griscti (griscti@acm.org)
%
% Technical questions _only_ to
% Gerald Murray (murray@hq.acm.org)
% ===============================================================
%
% For tracking purposes - this is V1.9 - April 2009

\documentclass{sig-alternate}

\begin{document}
%
% --- Author Metadata here ---
%\conferenceinfo{WOODSTOCK}{'97 El Paso, Texas USA}
%\CopyrightYear{2007} % Allows default copyright year (200X) to be over-ridden - IF NEED BE.
%\crdata{0-12345-67-8/90/01}  % Allows default copyright data (0-89791-88-6/97/05) to be over-ridden - IF NEED BE.
% --- End of Author Metadata ---

\title{General models for defect prediction?}
\subtitle{[progress report]}
%
% You need the command \numberofauthors to handle the 'placement
% and alignment' of the authors beneath the title.
%
% For aesthetic reasons, we recommend 'three authors at a time'
% i.e. three 'name/affiliation blocks' be placed beneath the title.
%
% NOTE: You are NOT restricted in how many 'rows' of
% "name/affiliations" may appear. We just ask that you restrict
% the number of 'columns' to three.
%
% Because of the available 'opening page real-estate'
% we ask you to refrain from putting more than six authors
% (two rows with three columns) beneath the article title.
% More than six makes the first-page appear very cluttered indeed.
%
% Use the \alignauthor commands to handle the names
% and affiliations for an 'aesthetic maximum' of six authors.
% Add names, affiliations, addresses for
% the seventh etc. author(s) as the argument for the
% \additionalauthors command.
% These 'additional authors' will be output/set for you
% without further effort on your part as the last section in
% the body of your article BEFORE References or any Appendices.

\numberofauthors{3} %  in this sample file, there are a *total*
% of EIGHT authors. SIX appear on the 'first-page' (for formatting
% reasons) and the remaining two appear in the \additionalauthors section.
%
\author{
% You can go ahead and credit any number of authors here,
% e.g. one 'row of three' or two rows (consisting of one row of three
% and a second row of one, two or three).
%
% The command \alignauthor (no curly braces needed) should
% precede each author name, affiliation/snail-mail address and
% e-mail address. Additionally, tag each line of
% affiliation/address with \affaddr, and tag the
% e-mail address with \email.
%
% 1st. author
\alignauthor
Lonnie Bowe 
%\titlenote{CS grad}
\\
       \affaddr{West Virginia University}\\
%       \affaddr{1932 Wallamaloo Lane}\\
%       \affaddr{Wallamaloo, New Zealand}\\
       \email{lbowe@csee.wvu.edu}
\alignauthor
Aglika Gyaourova \\      
       \affaddr{West Virginia University}\\
       \email{agyaouro@csee.wvu.edu}\\
\alignauthor
Zack Hutzell\\
       \affaddr{West Virginia University}\\
       \email{zhutzell@csee.wvu.edu}\\
}      

\maketitle
\begin{abstract}
\end {abstract}

\keywords{defect predictors, static attributes, general model}

\section{Introduction}
\section{Databases}


\section{Methods}
Our goal is to find a general model for defect predictors. For this purpose, the experiments are divided into  
within company (WC) and cross-company (CC). For the WC experiments, 90\% of one data set 
is used for training the classifier and the rest 10\% for evaluating the classifier performance. This process is 
repeated 10 times, by randomly selecting the instances to be used for evaluation. 

In all data sets, the distribution of the instances from the two classes, i.e. the defect and the 
non-defect, is imbalanced. The defect class is represented by a smaller number of instances 
 ranging  from 0.4\% to 32\% of all instances in a dataset. While training the number of instances of the 
 non-defect class was reduced to the number of instances in the defect class by random deletion. 

The Bayes classifier finds a decision that minimizes the Bayes error among classes. This is the 
theoretically optimal classifier when the attribute values of each class classes have a normal probability 
distribution. The Bayes classifier operates using the posterior probabilities of the classes as defined by 
the Bayes' theorem. In a two class problem having only a single attribute, the classifier decision is the 
attribute value at which the posterior probability of one of the classes becomes larger than the posterior 
probability of the other class. The naive Bayes classifier assumes independent attributes. 
Practice in software engineering has shown that the Bayes classifier performs well even when its 
assumptions are not met. 

Applying naive Bayes on categorical (discrete) data is fast and easy. 
When using categorical (or discrete) data, the naive Bayes operates on the conditional frequencies of 
the categorical values is fast and easy. Furthermore, discretization of continuous attributed does not 
have significant effect on the performance \cite{Dougherty95}.


\section{Experiments}
In the CC experiments, the training is performed using the 10 nearest neighbors of each instance in the 
evaluation set (without duplication). 

Minmax normalization.

Row deletion.

Feature selection.

\subsection{Performance evaluation}



% The following two commands are all you need in the
% initial runs of your .tex file to
% produce the bibliography for the citations in your paper.
\bibliographystyle{abbrv}
\bibliography{defects}  % sigproc.bib is the name of the Bibliography in this case
% You must have a proper ".bib" file
%  and remember to run:
% latex bibtex latex latex
% to resolve all references
%
% ACM needs 'a single self-contained file'!
%
%APPENDICES are optional
%\balancecolumns
\appendix
\end{document}

