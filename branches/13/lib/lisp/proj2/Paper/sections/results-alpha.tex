During testing it was determined that MultiPipes could easily be used 
as a reliable ???narrower??. We found that on some datasets we were 
able to acheive a very high reliability. That is, our resulting output 
set contained the actual class more than 90% of the time. However, we
found that on some other datasets the reliability was quite less. 
When searching for a solution we realized we needed a way to expand 
the returned dataset. This would increase the chance that the desired 
class is included in the returned set. The reason to expand the 
returned set is so we can improve the reliability to say with greater
confidence that the returned set is complete and accurate. Our next 
step is to train the learner as to which alpha value it should use. 
We would specify a goal reliability and expand the alpha slowly until 
that reliability is reached. If you look at Program 2 above you will 
notice that we add anything to the BestClass list that has a score 
equal to the best score found. Each score is a fraction and the best 
possible score is 1. When you apply an Alpha of .1 it basically 
states that if the the current score is greater than the current best 
score minus .1 append it to the list. This requires one run through the 
list of HyperPipes to find the best score. It then calculates the score 
to beat as \begin{math}ScoreToBeat = BestScore - Alpha\end{math}. On its second run through the 
HyperPipes any HyperPipe with a score greater than or equal to the 
\begin{math}ScoreToBeat\end{math} is included in the resulting set. 

Results of expanding alpha (graph)

Analysis of growth in enclosure with alpha changes

