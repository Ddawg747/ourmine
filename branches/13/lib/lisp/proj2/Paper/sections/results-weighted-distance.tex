There are two methods we use for calculating the distance from 
the mean in our algorithm. As stated above in "Fixing Outliers" 
we noted that an attribute range scores a 1 if it falls within 
the min and max for that attribute in a class. Our modification, 
as stated above was to use a normalized calculation to determine 
a value between 0 and 1 based on the attribute values distance 
from the mean attribute value in that class. Two calculations 
were discussed and they are as follows:
\subsubsection{Weighted Distance From Mean 1}
In this calculation it was decided that the distance from mean 
should be calculated as:
\begin{equation}
  DistFromMean=\frac{(Max-Min)-|(Mean-Val)|}{(Max-Min)}
\end{equation}
where Max, Min, and Mean are the max, min, and mean values for 
the attribute in the HyperPipe for the class currenlty being 
scored and Val is the current attribute value in the row being 
classified.
\subsubsection{Weighted Distance From Mean 2}
\begin{equation}
  LargestGap = max((Max-Mean),(Mean-Min))
\end{equation}
\begin{equation}
  DistFromMean=\frac{(LargestGap)-|(Mean-Val)|}{(LargestGap)}
\end{equation}
where Max, Min, and Mean are the max, min, and mean values for 
the attribute in the HyperPipe for the class currenlty being 
scored and Val is the current attribute value in the row being 
classified.



graph of weighted distance classification accuracy

