\subsubsection{Fixing Tied Classes}
When it was discovered that HyperPipes tended to choose classes 
towards the end of the classes list we investigated further to 
find that many classes were tieing against other classes. We 
then decided that this was unacceptable. Overwriting a class 
with another class of the same score puts a large emphasis on 
the order in which you score the classes. For this issue we 
decided to temporarily throw away the idea of classification 
and modified the code to return any classes who tied. This 
simple modification proved to us that this issue alone was a 
major factor in HyperPipes demise when put up against other 
classifiers. To recap the original hyperpipes said that if 
the current score is equal to or greater than the best score 
set the best class to the current class. After modification 
hyperpipes now states that if the score is greater than the 
best score set the best class to an array containing only the 
current class. If the current score is equal to the best score 
then append the current class to the list of best classes.
\subsubsection{Fixing Over Fitting}
As described previously Hyperpipes has an over fitting issue 
when it learns too much. While we have no implemented these 
potential fixes our possible solutions are described below:

\paragraph{Limit Number of Rows To Be Learned}
	It might be effective to simply limit the number of rows 
	that HyperPipes will use when doing its learning. This 
	number of rows might be calculated based on the number 
	classes and it may also require that this limit be 
	evenly distributed across all classes.
\paragraph{Detect Overfitting by Class Overlap}
	It may be possible to detect over-fitting by determining 
	the amount of overlap between classes. In other words, 
	if the number of attributes in a HyperPipe reaches a 
	certain level we could say that we should not modify our 
	HyperPipes with the information learned in this new line
	as it would cause too much overlap between classes.

\subsubsection{Fixing Outliers}
\subsubsection{Fixing Memory Management}
