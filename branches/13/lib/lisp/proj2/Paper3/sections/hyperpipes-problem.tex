\subsubsection{Tied Classes}
During development of the original HyperPipes. implementation it was noted 
that the code only kept track of the most recently seen class with a score 
equal to or greater than the largest score seen to that point. This means
that when running HyperPipes if multiple classes tied in score the class
tested last would be blindly chosen. We found that this anomoly caused our 
HyperPipes implementation to prefer the last classes discovered once it 
learns many rows.
\subsubsection{Susceptible to Over Fitting}
HyperPipes has the ability to have very low overhead costs in terms of 
memory. However, this benefit is overshadowed by HyperPipes susceptibility 
to overfitting. As more and more rows are added as experience to a HyperPipe 
its min and max values with slowly approach the min and max of the attribute 
alone regardless of class. This causes HyperPipes guessing ability to 
completely fall apart. As the min and max values for the attributes in every
HyperPipe expand the scores become very high and the likelyhood of 
encountering the issue in the previous sub-section becomes greater. That is, 
as each of the HyperPipes expand their min and max attribute values start to 
become so similar to eachother that they all begin to acheive common scores.
\subsubsection{Outliers Ruin Scoring}
As shown in the AddExperience function (Program \ref{pro:AddExperience}) a HyperPipe contains a set 
of bounds for each attribute. These bounds indicate the min and a max value 
ever seen for this attribute for this class. This becomes a problem when an 
outlier for that attribute occurs. Imagine after 10,000 rows your min is 23 
and your max is 45. You encounter a new row where this attributes value is 
431. This causes your new max value to be 431. This can be a major pitfall 
if this is the only time a number this large is encountered for this class. 
 You have now expanded this HyperPipe to a max so large that this attribute 
loses its ability to classify (It always matched on this attribute for this 
class).

