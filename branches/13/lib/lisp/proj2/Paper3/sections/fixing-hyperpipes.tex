\subsection{Introducing MultiPipes}
We saw incredible potential in HyperPipes if we could reign in
the issues described above. One of the major advantages of HyperPipes
is its ability to classify a single row very fast. The Pipes 
condition can be stored and retrieved at a later time for 
classification unlike some other learners whose classification 
relies heavily on the actual contents of previous rows. Below we 
describe how HyperPipes was transformed into MultiPipes.

\subsection{Fixing the issues}
\subsubsection{Tied Classes}
When it was discovered that HyperPipes tended to choose classes 
towards the end of the classes list we investigated further to 
find that many classes were tieing against other classes. We 
then decided that this was unacceptable. Overwriting a class 
with another class of the same score puts a large emphasis on 
the order in which you score the classes. For this issue we 
decided to temporarily throw away the idea of classification 
and modified the code to return any classes who tied. This 
simple modification proved to us that this issue alone was a 
major factor in HyperPipes demise when put up against other 
classifiers. To recap, HyperPipes states that if 
the current score is equal to or greater than the best score 
set the best class to the current class. MultiPipes states that if the score is greater than the 
best score set the best class to an array containing only the 
current class. If the current score is equal to the best score 
then append the current class to the list of best classes.


\subsubsection{Over Fitting}
As described previously Hyperpipes has an over fitting issue 
when it learns too much. All of the algorithms below are triggered 
based on the percentage of times the class is returned but it is 
found not to be the actual class. Two modifiable attributes, detectAfter
and triggerWhen, indicate to the system how often to check for 
overfit and at what percentage we assume we have hit an overfit. 
Below we describe the different actions we take when an overfit
is detected:
\begin{enumerate}
\item Reset and Relearn: 
	When this overfit action is enabled it will keep track of the 
	last 100 rows encountered for each class. If an overfit is detected 
	the numerical bounds for that class are reset to nothing(start over).
	However, it then looks at the oldest 50 rows in the 100 row history 
	and relearns those into the pipe before continuing. Therefore when
	an overfit is detected we dont lose all classification ability on that
	pipe, we simply know less.
\item Revert Pipe on Detection:
	When this overfit action is enabled it will keep track of the previous
	ten MultiPipes. When the supporting code requests that the MultiPipe
	be checked for overfit if it is determined to not be overfit the 
	pipe is logged. If the pipe is determined to be overfit it will revert 
	the pipe to the previously logged MultiPipe.
	
\end{enumerate}


\subsubsection{Outliers}
We have implemented some fixes to the problem of outliers. 
However, there are many other outlier fixing strategies that have 
not been implemented. Below is a list of our outlier fixing 
strategies and thier implemtation status:
\begin{enumerate}
\item Weighted Means: What this fix 
does is it changes the scoring mechanism. In the pseudocode for 
Classify (Program \ref{pro:Classify}) you will notice that it returns 1 if the current attribute 
value falls within the range of that given by the MultiPipe for that 
Attribute. The Weighted Means changes the 1 to a number between 0 
and 1 which represents the distance from the mean within the range. 
Two versions of this are explained further in Section 5.
\item Outlier Detection Algorithm: 
Every time MultiPipes attempts to add experience to its numerical bounds
the algorithm calculates the z-score\cite{Larsen01} for the value attempting to be added. 
If the Z-Score is greater than 1.96 then the value is not used to expand
the ranges within the hyperpipe.
\item Slowly Changing Ranges: 
The same fix we used in the outlier detection algorithm also allows for 
slowly changing ranges. Its important to realize that the value being
considered is added to the statistics for consideration in the z-score 
calculation and its addition to the statistics is not reversed if the
value is determined to be an outlier. This means that if the value 
determined to be an outlier occurs multiple times it is likely that 
it will eventually receive a z-score less than 1.96 and no longer be
considered an outlier and will subsequently be added to the min and max
for the hyperpipe.
\end{enumerate}

