\subsubsection{Centroid Assumption}
One method we implemented for classifying to a single class was
a centroid assumption function we created. The idea behind this
algorithm is that when a tie is formed we calculate the class 
which is believed to be at the center of the returned classes. 
This center class is determined by scoring each MultiPipe's 
attribute overlap with other MultiPipes. For each MultiPipe we 
sum the number of attributes in that MultiPipe whose min and max 
overlap with another classes min and max. In other words, for a 
single attribute in a MultiPipe the maximum score for that 
attribute is the number of MultiPipes minus 1. The maximum score 
for the MultiPipe as a whole is the number of attributes times 
the number of MultiPipes minus 1. After calculating this sum for 
each MultiPipe the class whose MultiPipe scores the highest 
is chosen as the classification. The results of this addition can 
be seen in Figure \ref{fig:performance}. The pseudocode for this 
algorithm can be found in Program \ref{PseudocodeCentroid}.
\begin{program}
\begin{algorithmic}
\Procedure {FindCentroid}{$HyperPipes$}
\State{$BestScore := 0$}
\ForAll{MainPipe in HyperPipes}
\State{$CurrentPipeScore := 0$}
\ForAll(TestPipe in HyperPipes)
\If{$MainPipe != TestPipe$}
\ForAll{Attr in MainPipe}
\If{$((MainPipe.Attr.min > TestPipe.Attr.min) \&\& (MainPipe.Attr.min < TestPipe.Attr.min)) || ((MainPipe.Attr.max < TestPipe.Attr.max) \&\& (MainPipe.Attr.max > TestPipe.Attr.max))$}
\State {$CurrentPipeScore++$}
\EndIf
\EndFor
\EndIf
\EndFor
\EndFor
\EndProcedure
\end{algorithmic}
\caption{MultiPipes FindCentroid Pseudo Code.}\label{PseudocodeCentroid}
\end{program}


