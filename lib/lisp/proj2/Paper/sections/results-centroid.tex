\subsubsection{Centroid Assumption}
One method we implemented for classifying to a single class was
a centroid assumption function we created. The idea behind this
algorithm is that when a tie is formed we calculate the class 
which is believed to be at the center of the returned classes. 
This center class is determined by scoring each HyperPipe's 
attribute overlap with other HyperPipes. For each HyperPipe we 
sum the number of attributes in that HyperPipe whose min and max 
overlap with another classes min and max. In other words, for a 
single attribute in a HyperPipe the maximum score for that 
attribute is the number of HyperPipes minus 1. The maximum score 
for the HyperPipe as a whole is the number of attributes times 
the number of hyperpipes minus 1. After calculating this sum for 
each HyperPipe the class within HyperPipe with the highest score 
is chosen as the classification. Pseudocode:
\begin{program}
\begin{algorithmic}
\Procedure {FindCentroid}{$HyperPipes$}
\State{$BestScore := 0$}
\ForAll{MainPipe in HyperPipes}
\State{$CurrentPipeScore := 0$}
\ForAll(TestPipe in HyperPipes)
\If{$MainPipe != TestPipe$}
\ForAll(Attr in MainPipe)
\If{$((MainPipe.Attr.min > TestPipe.Attr.min) && (MainPipe.Attr.min < TestPipe.Attr.min)) || ((MainPipe.Attr.max < TestPipe.Attr.max) && (MainPipe.Attr.max > TestPipe.Attr.max))$}
\State {$CurrentPipeScore++$}
\EndIf
\EndFor
\EndIf
\EndFor
\EndFor
\EndProcedure
\end{algorithmic}
\caption{HyperPipes Classify Pseudo Code.}\label{PseudocodeClass}
\end{program}
graph of centroid learning results

